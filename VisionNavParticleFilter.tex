
\documentclass[conference]{IEEEtran}
\usepackage{mathtools}
\usepackage{amsmath}
\usepackage{graphicx}
\usepackage{epstopdf}
\usepackage{lipsum}
\usepackage{float}
\usepackage{textcomp}
\usepackage{comment}
\usepackage{subcaption} 

% correct bad hyphenation here
\hyphenation{op-tical net-works semi-conduc-tor}


\begin{document}
%
% paper title
% Titles are generally capitalized except for words such as a, an, and, as,
% at, but, by, for, in, nor, of, on, or, the, to and up, which are usually
% not capitalized unless they are the first or last word of the title.
% Linebreaks \\ can be used within to get better formatting as desired.
% Do not put math or special symbols in the title.
\title{Coarse Geo-Registration Over A Large Area}


% author names and affiliations
% use a multiple column layout for up to three different
% affiliations

\author{\IEEEauthorblockN{Sean Lantto}
\IEEEauthorblockA{Mechanical and Aerospace Engineering Department\\
	Statler College of Engineering and Mineral Resources\\
	West Virginia University\\
	Morgantown, West Virginia 26505\\
Email: slantto@mix.wvu.edu}}
%\and

% conference papers do not typically use \thanks and this command
% is locked out in conference mode. If really needed, such as for
% the acknowledgment of grants, issue a \IEEEoverridecommandlockouts
% after \documentclass

% for over three affiliations, or if they all won't fit within the width
% of the page, use this alternative format:
% 
%\author{\IEEEauthorblockN{Michael Shell\IEEEauthorrefmark{1},
%Homer Simpson\IEEEauthorrefmark{2},
%James Kirk\IEEEauthorrefmark{3}, 
%Montgomery Scott\IEEEauthorrefmark{3} and
%Eldon Tyrell\IEEEauthorrefmark{4}}
%\IEEEauthorblockA{\IEEEauthorrefmark{1}School of Electrical and Computer Engineering\\
%Georgia Institute of Technology,
%Atlanta, Georgia 30332--0250\\ Email: see http://www.michaelshell.org/contact.html}
%\IEEEauthorblockA{\IEEEauthorrefmark{2}Twentieth Century Fox, Springfield, USA\\
%Email: homer@thesimpsons.com}
%\IEEEauthorblockA{\IEEEauthorrefmark{3}Starfleet Academy, San Francisco, California 96678-2391\\
%Telephone: (800) 555--1212, Fax: (888) 555--1212}
%\IEEEauthorblockA{\IEEEauthorrefmark{4}Tyrell Inc., 123 Replicant Street, Los Angeles, California 90210--4321}}




% use for special paper notices
%\IEEEspecialpapernotice{(Invited Paper)}




% make the title area
\maketitle

% As a general rule, do not put math, special symbols or citations
% in the abstract
\begin{abstract}
An aerial pla
\end{abstract}

% no keywords




% For peer review papers, you can put extra information on the cover
% page as needed:
% \ifCLASSOPTIONpeerreview
% \begin{center} \bfseries EDICS Category: 3-BBND \end{center}
% \fi
%
% For peerreview papers, this IEEEtran command inserts a page break and
% creates the second title. It will be ignored for other modes.
\IEEEpeerreviewmaketitle



\section{Introduction}
Many aircraft use a coupled inertial navigation and GNSS system to provide a attitude, position, and velocity estimates.The measurements from the inertial navigation system (INS) suffer from an unbounded error, that accumulates over time. This "drift" is corrected using measurements from the GNSS, integrated through a  Bayesian estimator such as a Kalman filter. An aircraft that has been denied GNSS positioning for an extend period will suffer from a very large error in its position, due to the INS drift. This large error means the aircraft may only know that it is within a large geographical region. A large geographical region is defined as ~60 x 60 kilometer area for this paper. Cameras on board the aircraft provide a method to localize the aircraft and correct for INS drift.\\ %TODO: false positives, converge in local area only
This work presents a georegistration based approach to localize an aircraft within a large geographic region using a particle filter. The particle filter is chosen to reduce the impact of false positive feature matches. Methods from [dons paper] are used for the creation of a database of georegistered features, as well as for matching features between the query image and the database. The particle filter uses the output from the feature matching to estimate the position of the aircraft as well as the field of view of the camera that produced the query image.

\section{Database creation}
%TODO: Search area is Dugway Proving Grounds, features converted from LLA position to NED, using one corner of search area as the origin
%TODO Need more detail on database creation
%TODO: include breif overview of SIFT, or reference to some one else
Scale invariant feature transform (SIFT) features are found on a satellite image of a geographic region. Each one of these SIFT features are given a descriptor (scale, size, magnitude, response, etc); as well as a WGS-84 location. The database features are sorted by the SIFT feature response value.

\section{Feature matching}
%TODO: Inculde description of FLANN?
%TODO: Include equations for weight calculation
SIFT features are generated for each image taken by the aircraft, and are sorted by response, just as the database features were. A Fast lookup for Approximation of Nearest Neighbor (FLANN) based matching algorithm is used to compare each SIFT feature generated in the query image to every feature in the database. This returns a list of the nearest neighbors, or the database features closest in appearance to the the query image feature being compared against. Using the L2 norm of the difference between the query image feature and the two nearest database features gives a distance between each neighbor. Taking the complement of the ratio of these distances, creates a weight that is assigned to the database feature that is the closest in appearance to the query image feature. These weighted georegistered database features are saved as a numpy array for use as the observation in the measurement model of the particle filter.

\section{Particle Filter}
The particle filter initializes with a uniform particle distribution over the entire geographic region. Each particle has a center location and a radius, which corresponds to the location and field of view respectively. Each particle is propagated between images using the measurements from the aircraft's INS. The estimated position is a weighted average of the particle centers. Weighting of the particles is discussed the measurement model section below.

\subsection{Measurement Model}


\subsection{Resampling and Particle Redistribution}

\section{Feel the need to put something here}

\section{Results}

\section{Conclusion and Future Work}
The conclusion goes here.


Can also be applied to full motion video data, that has unreliable metadata.

%TODO:Future Work: Increase number of states (vel, heading, etc), feature matching on the fly(only search area defined by particles for feature matches), robustize filter, other work


% conference papers do not normally have an appendix


% use section* for acknowledgment
\section*{Acknowledgment}


The authors would like to thank...





% trigger a \newpage just before the given reference
% number - used to balance the columns on the last page
% adjust value as needed - may need to be readjusted if
% the document is modified later
%\IEEEtriggeratref{8}
% The "triggered" command can be changed if desired:
%\IEEEtriggercmd{\enlargethispage{-5in}}

% references section

% can use a bibliography generated by BibTeX as a .bbl file
% BibTeX documentation can be easily obtained at:
% http://mirror.ctan.org/biblio/bibtex/contrib/doc/
% The IEEEtran BibTeX style support page is at:
% http://www.michaelshell.org/tex/ieeetran/bibtex/
%\bibliographystyle{IEEEtran}
% argument is your BibTeX string definitions and bibliography database(s)
%\bibliography{IEEEabrv,../bib/paper}
%
% <OR> manually copy in the resultant .bbl file
% set second argument of \begin to the number of references
% (used to reserve space for the reference number labels box)

\bibliographystyle{IEEEtran}
\bibliography{PF_bibtex_database}
%\bibliographystyle{aiaa}




% that's all folks
\end{document}


